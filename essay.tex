\documentclass[12pt]{article}

\usepackage[letterpaper]{geometry}
\usepackage{times}
\geometry{top=1.0in, bottom=1.0in, left=1.0in, right=1.0in}

\usepackage{setspace}
\doublespacing

\usepackage{rotating}

\usepackage{fancyhdr}
\pagestyle{fancy}
\lhead{}
\chead{}
\rhead{\thepage}
\lfoot{}
\cfoot{}
\rfoot{}
\renewcommand{\headrulewidth}{0pt}
\renewcommand{\footrulewidth}{0pt}
\setlength\headsep{0.333in}


\newcommand{\bibent}{\noindent \hangindent 40pt}
\newenvironment{workscited}{\newpage \begin{center} Works Cited \end{center}}{\newpage }


\begin{document}
\begin{flushleft}

Name: Max Bellomy\\
Professor: Zachary Callaway\\
Class: Humanities IV\\
November 17th, 2021\\

\begin{center}
Community, Family, and the Gulag Free Market
\end{center}


\setlength{\parindent}{0.5in}

%%%% Introduction (Funnel)

In the Gulag, conditions are harsh, sentences are harsher, and the work is brutal and unforgiving.
Prisoners are identified numerically, items permitted into the prison are strictly regulated and monitored, and almost no time at all is given for socialization or reflection.
Despite all of these restrictions, the prisoners' human nature and personal interests come together to form communities, exchange goods and services, and create something resembling a free market economy.

%%% Paragraph One (Yes, this is a free market economy.)
The Gulag's community is divided into prisoners, guards, and staff, and yet further into the prisoners' work squads, which are all in competition with one another.
"No wonder the squad leader looked so worried, that was his job-- to elbow some other squad, some bunch of suckers, into the assignment instead of the 104th."\footnotemark
The prisoners all share common interests, but due to their separation into groups, are naturally bound to compete over these interests rather than cooperate towards them.
Every squad is isolated in this sense, but the competition brings further nuance, such as adding purely profit-motivated trading between squads.
"He meant to buy tobacco at the price he'd paid before-- one ruble a glassful, though, outside, that amount could cost three times as much..."\footnotemark  
Tobacco is presumably smuggled in by guards and prison staff to be given to prisoners in exchange for money or favors, forming another direct connection between different parts of the Gulag community.
The prisoners have established their own economy, driven by what little money they're able to get from guards and prison staff


%%% Paragraph Two (Shukhov's place in the economy.)
Our protagonist, Shukhov, fits the role of your average blue collar worker.
"Shukhov did private jobs to get money, making slippers out of customers' rags-- two rubles a pair-- or patching torn jackets, price by agreement."\footnotemark
Even in the Gulag, people have things they need done, and things they can trade. 
National currency has found its way into the Gulag, acting just as it does in the outside world as a way to pay for anything. The Gulag does not operate on a barter system, but with a legitimate currency.
"So leave envy to the fellow who always think the radish in the other fellow's hand is bigger than yours. Shukhov knows life and never opens his belly to what doesn't belong to him."\footnotemark
The Gulag's economy is a limited one, and Shukhov understands that. 
Those who appear to have more than you will be forced to give up more of it as a result. To be envious of it is shortsightedness. 

%%% Paragraph Three (The Work-squadron, and how it is like a Family.)
The primary exception to the Gulag's free market system is within the work-squadron, which is seen to occasionally operate on trust and favors instead of hard currency.


\begin{center}
"He was on the point of leaving when he felt a twinge of pity for Tsezar. 
It wasn't that he wanted to make anything more of the man, he felt genuinely sorry for him... Pityingly, Shukhov gave him some advice: 
'Sit here till the last moment, Tsezar Malkovich... 
When the guard comes by the bunks with the orderlies and pokes into everything come out and say you're feeling bad. 
I'll go out first and I'll be back first. That's the way...'"\footnotemark
\end{center}

Shukhov has no material incentive to help Tsezar with keeping his parcels, but because he pities Tsezar and has the opportunity and ability to help.
By doing a favor for Tsezar without expecting direct compensation, Shukhov has earned Tsezar's trust and respect, which ultimately earns him compensation later down the line. 
"Alyosha returned. Impractical, that's his trouble. Makes himself nice to everyone but doesn't know how to do favors that get paid back. 'Here you are,' said Shukhov, and handed him a biscuit."\footnotemark
Within the squadron, just being a consistently good or nice person will win you favors. Alyosha didn't do anything for Shukhov to earn a biscuit, but merely offered meaningful conversation and reflection. That is enough to earn respect, and to receive help from your squadmates.
Shukhov is the perfect man for working and bartering for things, but doesn't understand the value of what Alyosha offers. Shukhov ends up giving Alyosha a biscuit perhaps out of pity, but more importantly out of respect. Shukhov pities Fetiukov too, but Fetiukov has never gained the squad's respect, and gets nothing.


%%% Conclusion (Reverse Funnel)
The various communities of the Gulag have created their own systems of trade, sometimes through favors, bartering, or monetary transaction. 
While it is true that for as long as two people each have something the other desires more, trade will develop, the complexity and scale of the Gulag's trade and community makes it a useful way to see how economies develop.
The Gulag is almost like an ant farm, or an aquarium. You have plenty of individuals that may not do very much on their own, but once together can form highly complex communities representative of systems far larger than themselves.



\begin{center}
Notes
\end{center}


\setlength{\parindent}{0.5in}

%%%% CITATIONS GO HERE.
1. A Day in the Life of Ivan Denisovich: Page 5 
2. A Day in the Life of Ivan Denisovich, Page 120
3. A Day in the Life of Ivan Denisovich, Page 120
4. A Day in the Life of Ivan Denisovich, Page 124
5. A Day in the Life of Ivan Denisovich, Pages 129-130 
6. A Day in the Life of Ivan Denisovich, Page 138
\end{flushleft}
\end{document}