\documentclass[12pt]{article}

\usepackage[letterpaper]{geometry}
\usepackage{times}
\geometry{top=1.0in, bottom=1.0in, left=1.0in, right=1.0in}

\usepackage{setspace}
\doublespacing

\usepackage{rotating}

\usepackage{fancyhdr}
\pagestyle{fancy}
\lhead{}
\chead{}
\rhead{\thepage}
\lfoot{}
\cfoot{}
\rfoot{}
\renewcommand{\headrulewidth}{0pt}
\renewcommand{\footrulewidth}{0pt}
\setlength\headsep{0.333in}


\newcommand{\bibent}{\noindent \hangindent 40pt}
\newenvironment{workscited}{\newpage \begin{center} Works Cited \end{center}}{\newpage }


\begin{document}
\begin{flushleft}

Name: Max Bellomy\\
Professor: Zachary Callaway\\
Class: Humanities IV\\
November 17th, 2021\\

\begin{center}
Community, Family, and their place in a Free Market
\end{center}


\setlength{\parindent}{0.5in}

%%%% Introduction (Funnel)

In the Gulag, conditions are harsh, sentences are harsher, and the work is brutal and unforgiving.
Prisoners are identified numerically, their ID number stitched to their uniforms, and there is very little time in the day for socializing or reflection.
Despite the dehumanizing treatment prisoners of the gulag receive, humanity's social nature persists. The prisoners still find time to socialize, and communities still develop.

%%% Paragraph One (Defining Community Structure)
(Statement) The community structure of the gulag is similar to that of a tree, branching from the Gulag as a whole into branches for prisoners and guards, and yet further into the prisoners' work groups.
(Evidence, Page 5) "No wonder the squad leader looked so worried, that was his job-- to elbow some other squad, some bunch of suckers, into the assignment instead of the 104th." <- Division and competition between the prisoners.
(Insight) The prisoners all share common interests, but due to their separation into groups, are naturally bound to compete over these interests rather than cooperate towards them.
(Insight) Every squad is isolated in this sense, but the competition brings further nuance, such as adding purely profit-motivated trading between squads.
(Evidence, Page 120) "He meant to buy tobacco at the price he'd paid before-- one ruble a glassful, though, outside, that amount could cost three times as much..."  
(Insight) The prisoners have established their own economy, driven by what little money they're able to get from guards and prison staff.
(Insight) Tobacco is presumably smuggled in by guards and prison staff to be given to prisoners in exchange for money or favors, forming another direct connection between different parts of the Gulag community.


%%% Paragraph Two (The Gulag Community as a Whole)
(Statement) As a whole, the Gulag is very similar to a free market economy.
(Evidence, Page 120) "Shukhov did private jobs to get money, making slippers out of customers' rags-- two rubles a pair-- or patching torn jackets, price by agreement."
(Insight) Even in the Gulag, people have things they need done, and things they can trade. 
(Insight) Money has even found its way into the Gulag, acting just as it does in the outside world as a standard universal currency. The Gulag does not operate on a barter system, but with a genuine currency.
(Evidence)
(Insight)
(Insight)

%%% Paragraph Three (The Work-squadron, and how it is like a Family.)
(Statement)
(Evidence)
(Insight)
(Insight)
(Evidence)
(Insight)
(Insight)

%%% Conclusion (Reverse Funnel)
(Specific Thing)
(Broader Extraction)
\begin{center}
Notes
\end{center}


\setlength{\parindent}{0.5in}

%%%% CITATIONS GO HERE.
\end{flushleft}
\end{document}