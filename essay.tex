\documentclass[12pt]{article}

\usepackage[letterpaper]{geometry}
\usepackage{times}
\geometry{top=1.0in, bottom=1.0in, left=1.0in, right=1.0in}

\usepackage{setspace}
\doublespacing

\usepackage{rotating}

\usepackage{fancyhdr}
\pagestyle{fancy}
\lhead{}
\chead{}
\rhead{\thepage}
\lfoot{}
\cfoot{}
\rfoot{}
\renewcommand{\headrulewidth}{0pt}
\renewcommand{\footrulewidth}{0pt}
\setlength\headsep{0.333in}


\newcommand{\bibent}{\noindent \hangindent 40pt}
\newenvironment{workscited}{\newpage \begin{center} Works Cited \end{center}}{\newpage }


\begin{document}
\begin{flushleft}

Name: Max Bellomy\\
Professor: Zachary Callaway\\
Class: Humanities IV\\
November 17th, 2021\\

\begin{center}
Observations on Community and Family in the Soviet Gulag
\end{center}


\setlength{\parindent}{0.5in}

%%%% Introduction

In the Gulag, conditions are harsh, sentences are harsher, and the work is brutal and unforgiving.
Prisoners are identified numerically, their ID number stitched to their uniforms, and there is very little time in the day for socializing or reflection.
However, the harshness of the conditions is not able to suppress human nature. The prisoners are still able to socialize, and communities still develop.

%%% Paragraph One: Defining the communities


%%% Paragraph Two: The general gulag community


%%% Paragraph Three: The work squadron, or the family


%%% Conclusion
\begin{center}
Notes
\end{center}


\setlength{\parindent}{0.5in}

%%%% CITATIONS GO HERE.
\end{flushleft}
\end{document}